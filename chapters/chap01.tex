\chapter{Beginnings}

\section{Introduction}

Today, artificial neural networks are a central building block of the machine learning landscape and are assumed to hold the biggest promise for the budding advent of true artificial intelligence. Neural Networks are integral to a wide array of applications as of the writing of this paper, such as image recognition, voice assistants, natural language processing and more. While debates over the safety and the implications of the looming AI singularity are entering the popular dialogue, most potential applications have not yet reached application in business as of yet.



Throughout the history of science, the inner workings of the human mind had been modeled along the most current understanding of outwardly applied science and engineering principles.

\section{predating 1900}

While in 335 BC Greek philosopher Aristotle assumed the brain to be a cooling mechanism for the blood with the seat of intelligence being the heart, the physician Galen posed in his \textit{balloonist theory} that nerves carried fluid as a signal and inflated muscles like balloons.

These from today's standpoint rather cartoonish conceptions of the nervous system gave way to the first indication that electricity played a part when in 1791 Luigi Galvani showed with his famous demonstrations of amphibian limbs that nerves carried electrical impulses from the brain. 

In 1906 the Nobel Prize in Physiology or Medicine was awarded to Camillo Golgi and Santiago Ramon y Cajal, \cite{golgi}, for their description of neurons as the building blocks of the brain.

\section{Mathematical modelling of neurons}


